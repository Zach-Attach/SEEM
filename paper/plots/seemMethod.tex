\begin{figure*}
  \centering
  \includegraphics[width=0.75\linewidth]{../data/images/other/SEEM_Method.png}
  \caption{\textbf{TODO: Change the Neural Ring to Nerve Ring} Visual depiction of the Spatially Embedded Edge Model (SEEM) algorithm. 
  For each neuron, SEEM finds the neuron whose soma is closest to it and opposite the Nerve Ring (i.e. the neuron's "penpal"). 
  The algorithm then draws a line segment in 3D space, representative of a neurite, originating in the neuron's soma, extending through the soma of the "penpal," and terminating at the boundaries of the body of \ce (blue lines in left side figure). 
  The boundaries are calculated as a range from the minimum coordinate - $\varepsilon$ to the maximum coordinate - $\varepsilon$ for each dimension ($x$,$y$,$z$). $\varepsilon$ is defined as the width of a neurite ($3 \mu m$). 
  If two of these "neurites" are $\le \varepsilon$ away from one another, the two neurons of these neurites form a potential connection (middle figure). 
  A random subset of these potential connections are chosen equal to the number of total connections found in the \ce network to which it is being compared (right figure).}
\end{figure*}