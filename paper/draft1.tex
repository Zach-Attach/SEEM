\documentclass[letterpaper]{article}
\usepackage{natbib,alifeconf}
\usepackage{hyperref}
\usepackage{import}
\usepackage{multirow, tabularx}
\usepackage{booktabs}
\usepackage{makecell}

\title{Spatial Embedding of Edges in a Synaptic Generative Model of \textit{C. elegans}}
\author{Zachary Laborde and Eduardo J. Izquierdo\\
\mbox{}\\
Cognitive Science Program, Indiana University Bloomington\\
Program in Neural Science, Indiana University Bloomington\\
zlaborde@iu.edu}


\begin{document}
\maketitle

\import{}{Sections/abstract}

\import{}{Sections/intro}

\import{}{Sections/methods}

\import{}{Sections/results}

\section{Discussion and Conclusions}
{\bf  What does it mean?} Start off with a brief summary of the results and the main insights gained. Are there any things that came up during experiments and interpretation of the Results that warrant a further discussion. What are the limitations of your model. You are encouraged to be a good critic of the work and express honestly its shortcomings. End in a forward-looking note. What could be done next? What do you see as the most immediate next steps that you would do (or that somebody else reading your paper might want to set out to examine). 

{\bf How long should this document be?} It doesn't matter for now. If you end up with a two-page report, that's fine. If you end up with a longer one, that's fine too. Keep in mind that the point of this report is to efficiently communicate your scientific findings in a way that makes your claims supported by clear evidence, your methods reproducible, and your results easily interpretable. 

{\bf Additional reading?} Finally, take some time to do some additional reading on the structure of scientific papers. These are two resources that I think might be useful (they are both hyperlinks, so should should be able to click on them): 
\href{https://www.nature.com/scitable/topicpage/scientific-papers-13815490/}{"Scientific Papers by Nature"} and
\href{https://abacus.bates.edu/~ganderso/biology/resources/writing/HTWsections.html}{"The Structure, Format, Content, and Style of a  Journal-Style Scientific Paper"}. 

\footnotesize
\bibliographystyle{apalike}
\bibliography{bibliography}

\end{document}
