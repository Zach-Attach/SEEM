\section{Results}

To answer the question of whether spatially embedding edges is an important factor in the formation of the \ce Neural Network (CENN), we compared networks from our initial model (SEEM) with \ce. 
We wanted to minimize the number of assumptions made in the model. 
Our assumptions in the model are the maximimum connecting distance between two neurites $\varepsilon$ is $3 \mu m$, that the filling fraction does not vary between any two neurons, and that neurites form in straight lines angled towards their "penpal" (see \textit{Methods}).
To account for the last assumption, we compared these results with a random direction version of our model (REEM). 
We also compared these networks with the Random Distance Dependent Attachment Model (RDDAM) from \cite{Itzhack} and a set of random networks (Erdos-Renyi Networks / ERN) to gauge how close our model is to the \ce network. 
% TODO: Explain what L1 and L5 are
To rule out development as a factor, we made comparisons to both a single L1 connectome ($\sim 0$ hours) and a set of 3 adult (L5) connectomes. 
We measured these generative network models from a set of 100.

\subsection{Network Statistics}
Comparing these networks can be done in several different ways. As an initial comparison, we chose to use network statistics to compare essential aspects of these graphs. We chose to compare the average clustering coefficient, average edge distance, average connectivity, and average total number of bidirectional links. These network statistics were chosen as they were used by \cite{Itzhack} in their paper on the RDDAM, providing an initial point of comparison with previous work. We plotted the results of the network measures of all 100 instances of each model (see fig 2) and compared them with the results of these measures on the CENN. For the newborn (L1), we compared it to a single graph, but, for the adults (L5), this was three graphs.

\import{plots}{statTableL1}
\import{plots}{statTableL5}  

In general, there were not many significant differences between the results of L1 and L5 networks. One difference of interest was that the networks of \cite{Witvliet} had noticeably different clustering coefficients and total number of bidirectional links when compared with the N2U connectome \citep{Durbin}.
It is difficult to determine what might be causing this discrepancy, whether it is a result of different environmental conditions or if this is indicative of flaws in recording methods of some or all of the connectomes. Rather than averaging the results of the connectomes, we chose to show all three individually (see fig 2). 
\import{plots}{statViolin}
\import{plots}{degreeDist}
\import{plots}{distDist}

\import{Sections/results/stats}{clustering}
\import{Sections/results/stats}{distance}

\import{Sections/results/stats}{connectivity}
\import{Sections/results/stats}{bidirectionalLinks}
\import{Sections/results/stats}{outro}

\subsection{Distributions}

While these statistics give us an idea of how some of the attributes of these networks in describing the underlying structure of the CENN, it does not give us a picture of how the edges in any of these networks are distributed. 
Network distributions can give us a better insight into the shape of these networks.
We had two questions: 
1) What is the structural makeup of these networks? (Are they scale-free, small-world, etc.?) 
2) What is the physical makeup of these networks?

To answer these questions, we compared the degree distributions of these networks to examine their network structure (see fig 3) and the distance distributions of these networks to examine their physical structure (see fig 4). 
To get a better idea of how similar these distributions were, we also measured their Wasserstein Distances from the \ce distribution. 
For the adults, these distributions were averaged.

\import{Sections/results/distributions}{degree}
\import{Sections/results/distributions}{distance}

\import{plots}{spatialFront}

\subsection{Spatial Embedding} % TODO
Examining the makeup in Euclidean space, we observe varying connection patterns between each model (see fig 5). 
Our random examples, ERN and REEM, show to not appear to have any skewed distribution in their preferred connections. 
On the other hand, our model (SEEM) and RDDAM paint a different picture. 
SEEM appears to prefer connections that do not span the center (pharynx). 
RDDAM appears to have connections mostly localized to the most clustered areas.
From these observations, \ce frontal ganglia forms connections around the entire pharynx more often than our model would expect.
Given that our model is unable to ``wrap'' around the pharynx, such a result is unsurprising.

\subsection{Overlaps} % TODO
Looking at the edges shared by the models and \ce, we find that none of the models' resulting networks contain a significant portion of the edges found in \ce (see fig 6). 
These results are unsurprising given the results of their average connectivity and their spatial embedding.
In spite of the network similarities that SEEM has in common with the \ce frontal ganglia, it is not an accurate approximation of it.

\import{plots}{overlaps}