\section{Conclusion}

While significant experimental progress is made to understand the developmental process of \ce, a computational modeling approach allows us to test what assumptions in our thinking are more and less important for the process. 
To date, there’s only been one model of the synaptogenesis of the \ce connectome that has aimed at predicting the connections between its neurons. 
In their work, they based their predictions on the distance between neurons alone. 
In this work, we found that embedding neurites are an important factor in explaining certain network aspects of the \ce frontal ganglia.
Although geometry is an important factor in determining connection in \ce, it is not the only factor that matters.
All models tested failed to effectively replicate average connectivity, implying these biological neurons create networks that are much more metabolically efficient than typical networks.
The results of spatial embedding and comparing edge overlaps show that despite the similarities our model has with the \ce frontal ganglia they are not representative of the \ce neuronal network.