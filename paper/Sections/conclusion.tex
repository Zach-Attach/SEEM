\section{Conclusion}

While significant experimental progress is made to understand the developmental process of C. elegans, a complementary modeling approach allows us to test what assumptions in our thinking are more and less important for the process. 

To date, there’s only been one model of the synaptogenesis of the C. elegans connectome that has aimed at predicting the connections between its neurons. In their work, they based their predictions on the distance between neurons alone. 

In this work, we build on that work and test one important hypothesis: whether neurites directed in space is predictive of the neural structure of C. elegans.

This implies that the geometry of C. elegans is the most significant factor in determining connections.