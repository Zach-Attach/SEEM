\section{Introduction}
Despite tremendous computational power, no person has been able to create an algorithm as capable as the human brain. With its approximately 86 billion neurons (Azevedo et al., 2009), the human brain is an organ so complex that charting its synaptic connections alone is a monumental task (Van Essen et al., 2012). Tremendous time and effort are being invested in the creation of a complete human connectome without a definitive explanation of how this data would explain how the human brain functions (Ceylan et al., 2022; Rheault et al., 2020). Accurate mapping of these connections in living brains alone prove difficult given modern technological limitations (Rheault et al., 2020). With a neuronal network that is approximately one percent of one percent the size of a human’s (Azevedo et al., 2009; White et al., 1986), Caenorhabditis elegans is the only organism today with a completely mapped connectome (White et al., 1986). One might expect that the neural functions of C. elegans would be well understood. However, the dynamics and behaviors of the C. elegans neuronal network (CENN) have not been accurately replicated. To truly understand how a brain works, a neuronal model replicating the structure and dynamics of said brain is essential (Izquierdo, 2019). 
Synaptic connections are a fundamental part of microscale connectomes (Sporns et al., 2005), but many of these connections vary dramatically across individuals. Over half of all synaptic connections within an adult C. elegans are variable (Witvliet et al., 2021). While the topology of synaptic connections cannot predict the dynamics of a neuronal network, it does restrict what sort of dynamics are possible (Prinz et al., 2004). Any computational model of C. elegans that cannot account for all potential variabilities within the CENN would not be biologically accurate. The CENN exhibits consistent global attributes despite having so many variations (Pérez-Escudero \& Polavieja, 2007). What combination of variable connections lead to biologically accurate CENN remains unclear (Witvliet et al., 2021). Any computational model of the CENN cannot be tested for complete accuracy without this information. 
Knowing what neuronal network topologies are possible for a neurotypical C. elegans is necessary to accurately modeling the entire dynamical CENN. Manually mapping all possible topologies of C. elegans leads to a combinatorial explosion. With 302 neurons, there are 302302 or ~9 × 10748 possible connection configurations for the CENN. This only becomes worse with more complex organisms such as Drosophila or humans. A trial-and-error approach is impractical in this context.
Generative models are a promising solution to this problem of network combinatorial explosion. Generative network models create networks from a set of rules defining where and how connections are created (Betzel \& Bassett, 2017). By focusing on how networks form and not on a complete list of nodes and edges, generative models are very scalable. These models are an effective solution for neuronal networks of any complexity. 
Several generative models attempt to accurately predict the network topology of C. elegans (Costa et al., 2007; Itzhack \& Louzoun, 2010; Khajezade et al., 2019; Nicosia et al., 2013). The Random Distance Dependent Attachment Model (RDDAM) is one such model (Itzhack \& Louzoun, 2010). It randomly forms connections with a probability that logarithmically decreases with an increase in spatial distance between two neurons. The RDDAM creates networks shown to resemble the CENN in average length of connections, average connectivity, and shortest path length distribution. Given its simple rules that align with observed neuronal organization in C. elegans (Pérez-Escudero \& Polavieja, 2007), the RDDAM provides an excellent framework for future generative models of CENN connectivity to which to compare or improve. An essential aspect of the RDDAM is the spatial embedding of each neuron. Neurons exist in physical space, interacting chemically and electrically with their surroundings. What connections they make are a result of their location relative to other cells (Hentschel \& Ooyen, 2000; Kaiser \& Hilgetag, 2006; Pérez-Escudero \& Polavieja, 2007). 
Spatial embedding is not only relevant for the somas but for the neurites as well. Neurites take up physical space within the body of C. elegans, limiting how many connections are physically possible within the organism. The neurons of C. elegans each have one or two neurites which rarely branch, forming most of their connections as “en passant” synapses (i.e. synapses that do not form at the tip of the axon) (Durbin, 1987; White et al., 1986). This pattern of neuronal connectivity is not accounted for in existing models of C. elegans neuronal connectivity (Costa et al., 2007; Itzhack \& Louzoun, 2010; Khajezade et al., 2019; Nicosia et al., 2013). Constraining connectivity to intersections along the path of a neurite may result in a network which is more similar to CENN than the results of existing generative model. I hypothesize that a generative model which spatially embeds both nodes and edges will result in a network more similar to CENN than RDDAM.
