\section{Model and Methods}
\subsection{NOTES}

The Model/Method needs an initial paragraph that describes the section more broadly and says more about how it will be organized. From reading the Method, it is really hard to understand what you are doing. Try to Zoom out first and explain the larger goal first. Then explain the algorithm broadly. Then walk through details one by one. Use steps and orderings to help you at some point: There are 5 key steps in the process of achieving Y. First, we do this. Then, this. Finally, this.

We sought to understand how spatial embedding of neurites might explain the neural organizational structure of \ce.We developed a model to create sets of networks which attempt to recreate the process of neurite growth and synapse formation in euclidean space. Our aim with this model is examine how neurites impact the structure of a network while keeping the number of assumptions to as few as possible.

Rather than examine the entire neuronal network of \ce, we chose to narrow our analysis to the frontal ganglia. This was chosen because many of \ce's motor neurons form connections in a more prescriptive fashion, leading to neurites much longer than those found in the frontal and posterior ganglia (CITE). In total 131 neurons were used as nodes for our networks.




\subsection{Data}
The length and organization of neural processes in C. elegans are highly variable, particularly in the pharyngeal nervous system. To simplify this, this research only used neurons in the frontal ganglia, whose length in connections are less variable. All synaptic connections of 131 neurons in the adult C. elegans frontal ganglia were downloaded from www.dynamic-connectome.org/resources (Kaiser \& Hilgetag, 2005; Kötter, 2004; McCormick et al., 2004). 3D positions of the neurons were taken from Skuhersky, Wu, Yemini, Boyden, \& Tegmark (2021). Newborn (L1) (~0 hours) connectome data was downloaded from https://nemanode.org (Witvlietet al., 2021)
\subsection{Model}
UPDATE WHEN ADDING RANDOM DIRECTION SEEM I developed a generative model which spatially embeds edges as well as nodes. This modelwill be referred to as the Spatially Embedded Edge Model (SEEM). SEEM iterates over every node. Each node locates the node closest to it that is on the opposite side of the nerve ring, a region which contains more than half of all neural processes (Altun, 2017). A line segment is drawn from the source node through the target node, terminating at the surface of the bounding box of the network. The bounding box is defined as the smallest 3D box that can hold all nodes in the network. These line segments are intended to represent a single neurite. After all “neurites” have been drawn, the overlaps between these line segments are recorded. Given these neurons have a diameter of 2-3 microns (Schafer, 2006), a distance less than or equal to 3 microns between any two lines will be considered an overlap.All possible edges in the model have now been determined. To generate a network, a random selection of directed edges are selected from the set of all possible edges. The number of edges selected can be adjusted. For these experiments, I used as many edges as were in the comparable CENN. The code implementing this algorithm can be found at the link listed under \textit{Data and Code Availability}. INSERT RANDOM
To better minimize the nunber of assumptions in our model, we created a second model for comparison. The Randomly Embedded Edge Model (REEM) is created in the same way as our other model (SEEM) with one key difference. Its neurites form in a random direction. This model can explain our algorithm for determining neurite direction influences the results.
\subsection{Network Comparison Methods}
The output of SEEM was compared to the CENN, RDDAM outputs, and Erdos-Renyi random graphs. Excluding CENN, each model was run for 1000 iterations, resulting in 1000 possible graphs. Each metric was run over every instance of the graph and the average value was recorded. 
To compare these networks, I used three measures used in Itzhack and Louzoun (2010). These were average clustering coefficient, average distance, and total bidirectional links. I attempted to include Average Connectivity as was done in the mentioned paper, but this proved difficult, so it was not included in this analysis. To better understand the connectedness of the graphs, I also included average closeness centrality as a fourth comparison. Average degree distributions were also included for comparison. Finally, visual comparisons of randomly selected spatially embedded graphs were also incorporated to highlight patterns in the resulting networks.
