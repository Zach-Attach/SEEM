\subsubsection{Edge Distance:} 

The average edge distance found that our model was the closest to \ce in this regard. 
It is important to note that our random model (REEM) and random networks (ERN) were not that far behind. 
RDDAM had much shorter edges on average than any of the other networks. 
Given that it prioritizes proximity when making connections, this result is unsurprising. 
Our random model (REEM) had edge distances which were slightly longer. 
It is unclear why this is the case. 
One possible explanation is that the neurites of this model was less likely to cross paths with their neighbors given the random directions they take.
Since both of our models were most similar to \ce in edge distance, this suggests that spatial embedding of neurites and soma alone mostly explains this characteristic of the network. 

\textbf{MOVE BACK}
Compared with the results from \cite{Itzhack}, ranging from 3-4 microns, our results are noticeably higher, ranging from 6-20 microns. 
The increase in distance can likely be explained by the dimensionality of the embedded space. 
Given that our tests incorporated three dimensional data of soma body locations from \cite{Skuhersky} while \cite{Itzhack} only used the one dimensional position data from \cite{WormAtlas2009}, the distance in our analysis between any two soma will always be equal to or greater than what one would find using one dimensional positions.

