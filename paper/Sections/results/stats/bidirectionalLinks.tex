\subsubsection{Bidirectional Links}
The number of bidirectional links in \ce were shown to be most similar to the number found in RDDAM.
Given how accurate it is in predicting this metric, it is likely that spatial proximity explains for the number of bidrectional edges in \ce.

It is interesting to note that these models have fewer bidirectional links in L1 and more bidirectional links in L5 when compared with \ce. 
Given that there are a greater number of connections in the L5 networks, more bidirectional links should be expected just from random chance alone, as we see in the results of our Erdos-Renyi networks. 
Although, the results of Erdos-Renyi networks show that random chance alone cannot explain amount of growth in connection between the two ages. 

Another simple explanation for this growth could be a result of the smaller number of connections possible under our model (see fig 1). 
One is more likely to flip two heads with a coin than one is to roll two 1's with a die, given the same number of flips/rolls. 
Our random model (REEM), on average found fewer potential connections compared with our initial model (SEEM). 
Given RDDAM's preference for short connections, an even smaller subset of possible connections are most likely to be made in the model, which are direction agnostic. 
This would explain why RDDAM has so many bidirectional links. 
These results show that spatial proximity of soma alone can explain for the network structure of \ce.