\subsubsection{Degree Distributions} 
We found that our models and RDDAM have degree distributions closest to \ce, with RDDAM being slightly closer. 
These models result in networks that appear to have more variability in degree than random networks. 
Since both rely on the spatial proximity of nodes, which are not evenly distributed, it would be expected for their degree distributions to favor nodes that are closer to many others. 
This could explain why we see these results. 
Notably, the degree distributions of CENN have much longer tales. 
This means that \ce connectomes are more centralized than any of the other networks 
This falls in line with the results measured from Average Connectivity. 