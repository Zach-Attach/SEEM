\section{Future Directions}
The results of this study have resulted in many potential avenues for future research.
Given the results of average connectivity, it would interesting what sort a network would look like that is able to capture this network attribute of \ce.
A simple heuristic to locally enforce a low local connecitivity in each node could be one way in which this could be possible.
It is unclear what sort of biologically realistic mechanism could explain for such an effect.

Another potential direction for this work would be to incorporate more biological mechanisms to the system.
One option would be to change the connectome during development. 
For example, adding nodes to the network over time or adjusting the node locations as the worm grows. 
It might also be interesting to add genetic factors in determining connectivity patterns. 

% Further analysis of the current graphs would also be important. For example, 
% % adjusting alpha in RDDAM

% flattening space for SEEM

% incorporating genetics

% incorporating development

% incoroporate bilateral symmetry or homophilic attachment

% compare directionality of electrical and chemical synapses (see bidirectionality)

% Plot the local connectivity data of all networks to better understand the discrepancy.

% A model that takes into account the flow of information or the efficiency of information may lead to a more similar graph.

% \textbf{NOTE:} I need to reexamine the source data to understand if these connections are electric or chemical. Shouldn't we expect the majority of connections to be bidirectional since most are electrical? Is there a misinterepretation of the data?

% We also implement RDDA in 3D space using new data.