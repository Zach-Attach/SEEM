\section{Future Directions}

The results of this study have resulted in many potential avenues for future research.
None of the models presented could replicate the average connectivity found in the \ce frontal ganglia. 
It would be interesting to see how a generative model could replicate this network attribute.
A simple heuristic to locally enforce low local connectivity in each node could be one way in which this could be possible.
It is unclear what biologically realistic mechanism could explain such an effect.
A generative model with low connectivity might further uncover the underlying mechanisms of the \ce neuronal network.

While these generative models have proved useful in understanding \ce, it is unclear if they can be generalized across species. 
Applying these models to the larval microconnectomes of \textit{Ciona intestinalis} \citep{Ryan}, \textit{Platynereis dumerilii} \citep{Veraszto}, and \textit{Drosophila melanogaster} \citep{Winding} would reveal whether these network features are species-specific.

Another potential direction for this work would be to incorporate more biological mechanisms into the system.
One option would be to change the connectome during development. 
For example, adding nodes to the network over time or adjusting the node locations as the worm grows. 
It might also be interesting to add genetic factors in determining connectivity patterns. 

% Further analysis of the current graphs would also be important. For example, 
% % adjusting alpha in RDDAM

% flattening space for SEEM

% incorporating genetics

% incorporating development

% incoroporate bilateral symmetry or homophilic attachment

% compare directionality of electrical and chemical synapses (see bidirectionality)

% Plot the local connectivity data of all networks to better understand the discrepancy.

% A model that takes into account the flow of information or the efficiency of information may lead to a more similar graph.

% We also implement RDDA in 3D space using new data.