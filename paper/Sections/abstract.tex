\begin{abstract}
  Despite so much research being put into neuronal networks, the development of their structure remains unclear. Synapses form because of many factors including spatial location, guidance molecules, and gene expression. It remains unclear how significant a role each of these factors plays in these networks’ formation. With a relatively small and well-understood connectome, C. elegans is an ideal subject for this topic. Many unknowns and experimental challenges currently impede a detailed understanding of nervous system development. Computational modeling of these processes can begin to answer these questions, potentially leading to new insights that would otherwise be difficult to attain. Previous research has attempted this by stochastically assigning connections, weighted by distance. This model created networks that were found to be, on average, within one standard deviation of an adult C. elegans connectome in distance, connectivity, and the number of bidirectional links. The model did, however, result in networks with a very different clustering coefficient, which was 2 standard deviations away from the C. elegans connectome. We found that this model created networks with degree distributions more akin to random networks than to a C. elegans connectome. While the spatial location of the somas was accounted for, the spatial location of the neurites was not. Neurons in C. elegans typically have 1-2 processes. This physically constrains where synapses occur. Spatially embedding these processes may result in networks more like a C. elegans connectome than in the past. We created a new model which forms connections at intersections of artificial neurites, which pass through the densest region of the nervous system, the nerve ring. Because sensory/motor neurons often develop in a predetermined manner, testing was restricted to the frontal ganglia. The model’s results were compared with the results of the previously mentioned model, random networks, and two C. elegans connectomes. The clustering coefficient, average distance, and total bidirectional links were all compared for consistency with previous research results. These metrics found the new model to be most like the C. elegans connectomes. The degree distributions of the resulting networks of this new model were closest to C. elegans. Mapping these networks in physical space revealed that the networks from this new model were visually dissimilar to those of C. elegans. Connections between neurons on opposite sides of the pharynx (the center) were underrepresented. The improvements made by this new model suggest spatial constraints of neural processes are important in the development of synaptic connections.
\end{abstract}