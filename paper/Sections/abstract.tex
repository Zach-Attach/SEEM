\begin{abstract}
  The human brain is poorly understood. 
  Although insufficient, investigating its structure is necessary to discern how it operates.
  This structure on a microscale can vary wildly between individuals.
  Understanding how these networks form would help in explaining this variability.
  To do so, we need to develop computational models that simulate the processes involved.
  With a relatively small and (near) completely reconstructed connectome, \ce is an ideal subject for this research.
  A previous attempt at this used stochastic methods, where connections are assigned randomly and weighted by the distance between soma. 
  While useful, this model failed to predict particular network attributes of the \ce connectome.
  We aimed to develop a minimal model that incorporates the spatial embedding of neurites to approximate the process of neurite growth and synapse formation in Euclidean space, examining the impact of neurites on network structure.
  We found that networks that incorporate the spatial embedding of neurites resulted in particular attributes consistent with connectomes of \ce.
\end{abstract}